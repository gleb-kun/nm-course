\documentclass[12pt]{article}

\usepackage{ucs}
\usepackage[utf8x]{inputenc}
\usepackage[russian]{babel}
\usepackage{array,graphicx,dcolumn,multirow,abstract,hanging,fancyhdr,float}

\usepackage{amssymb}
\usepackage{amsmath}
\usepackage{indentfirst}

\DeclareMathOperator{\cond}{cond}
\newcommand{\norm}[1]{\left\| #1 \right\|}


\begin{document}
    \begin{flushright}
    {v0.4.2}
    \end{flushright}
    \begin{flushright}
	{01.09.2025}
	\end{flushright}
    \section*{Вычислительный практикум}
    \section*{Практическое занятие \#7.1. Метод Гаусса}

    \subsection*{Общая информация}

    Рассмотрим процедуру метода Гаусса в общем виде. Дана система линейных алгебраических уравнений (СЛАУ) порядка~$n$
    \begin{equation}
        \begin{gathered}
        	\begin{cases}
        		a_{11}^{\left(0\right)}x_1+a_{12}^{\left(0\right)}x_2+\ldots+a_{1n}^{\left(0\right)}x_n=b_{1}^{\left(0\right)}\\
        		a_{21}^{\left(0\right)}x_1+a_{22}^{\left(0\right)}x_2+\ldots+a_{2n}^{\left(0\right)}x_n=b_{2}^{\left(0\right)}\\
        		\ldots\\
        		a_{n1}^{\left(0\right)}x_1+a_{n2}^{\left(0\right)}x_2+\ldots+a_{nn}^{\left(0\right)}x_n=b_{n}^{\left(0\right)}
        	\end{cases}.
        \end{gathered}\label{7.1:f1}
    \end{equation}

    Предположим, что коэффициент $a_{11}^{\left(0\right)}$, называемый \emph{ведущим элементом} первой строки, не равен нулю.

    Найдем величины
    \begin{equation}
        \mu_{i1} = \frac{a^{\left(0\right)}_{i1}}{a^{\left(0\right)}_{11}},\quad i=2,3,\ldots, n.\label{7.1:f2}
    \end{equation}
    

    Далее исключаем неизвестную $x_1$ из каждого уравнения системы, начиная со второго. Для этого вычитаем из второго, третьего$,\ldots, n$-го уравнения системы (\ref{7.1:f1}) первое уравнение умноженное на коэффициент $\mu_{21}$, $\mu_{31},\ldots, \mu_{n1}$ соответственно. Преобразованная система уравнений имеет вид
    \begin{equation}
    	\begin{cases}
    		a_{11}^{\left(0\right)}x_1+a_{12}^{\left(0\right)}x_2+a_{13}^{\left(0\right)}x_3+\ldots+a_{1n}^{\left(0\right)}x_n=b_1^{\left(0\right)}\\
    		\hspace{15.3mm} a_{22}^{\left(1\right)}x_2+a_{23}^{\left(1\right)}x_3+\ldots+a_{2n}^{\left(1\right)}x_n=b_{2}^{\left(1\right)}\\
    		\hspace{15.3mm}\ldots\\
    		\hspace{15.3mm} a_{n2}^{\left(1\right)}x_2+a_{n3}^{\left(1\right)}x_3+\ldots+a_{nn}^{\left(1\right)}x_n=b_{n}^{\left(1\right)}\\
    	\end{cases},
    	\label{7.1:f3}
    \end{equation}
    где
    \begin{equation}
        a_{ij}^{\left(1\right)}=a_{ij}^{\left(0\right)}-\mu_{i1}\cdot a^{\left(0\right)}_{1j},\quad b_{i}^{\left(1\right)}=b_{i}^{\left(0\right)}-\mu_{i1}{b^{\left(0\right)}_{1}},\quad i,j=2,3,\ldots,n.\label{7.1:f4_1}
    \end{equation}
    Поступаем аналогично со следующим уравнением из преобразованной системы. В конечном итоге приводим исходную систему к эквивалентной системе
    \begin{equation}
    	\begin{cases}
    		a_{11}^{\left(0\right)}x_1+a_{12}^{\left(0\right)}x_2+a_{13}^{\left(0\right)}x_3+\ldots+a_{1n}^{\left(0\right)}x_n=b_1^{\left(0\right)}\\
    		\hspace{15.3mm} a_{22}^{\left(1\right)}x_2+a_{23}^{\left(1\right)}x_3+\ldots+a_{2n}^{\left(1\right)}x_n=b_{2}^{\left(1\right)}\\
    		\hspace{15.3mm}\ldots\\
    		\hspace{47.5mm} a_{nn}^{\left(n-1\right)}x_n=b_{n}^{\left(n-1\right)}\\
    	\end{cases}
    	\label{7.1:f4}
    \end{equation}
    с верхней треугольной матрицей $\bold{A}^{\left(n-1\right)}$.
    
    Далее обратным ходом из системы (\ref{7.1:f4}) последовательно находим неизвестные $x_n$, $x_{n-1}$, $\ldots$,~$x_1$.

    \subsection*{Число арифметических операций в методе Гаусса}

    Узнаем количество арифметических операций, выполняемых при решении СЛАУ порядка $n$.

    Вычисления 1-го шага прямого хода с использованием формул (\ref{7.1:f2}) и~(\ref{7.1:f4_1}) требуют выполнения $n-1$ делений, $\left(n-1\right)n$ умножений и $\left(n-1\right)n$ вычитаний, таким образом общее число арифметических операций, выполняемых на первом шаге, равно
    \begin{equation}
        Q_1=2\left(n-1\right)^2+3\left(n-1\right).
    \end{equation}
    Аналогично, на втором шаге требуется операций
    \begin{equation}
        Q_2=2\left(n-2\right)^2+3\left(n-2\right).
    \end{equation} 
    Не трудно догадаться, что для исключения неизвестной $x_i$ надо затратить операций
    \begin{equation}
        Q_i=2\left(n-i\right)^2+3\left(n-i\right).
    \end{equation}

    Подсчитаем общее число необходимых арифметических операций в прямом ходе
    \begin{equation}
        \begin{gathered}
            Q_{\text{forward}}=\sum^{n-1}_{i=1}Q_i=2\sum_{i=1}^{n-1}\left(n-i\right)^2+3\sum_{i=1}^{n-1}\left(n-i\right)=2\sum_{i=1}^{n-1}i^2+3\sum_{i=1}^{n-1}i=\\=\frac{2\left(n-1\right)n\left(2n-1\right)}{6}+\frac{3\left(n-1\right)n}{2}=\frac{2}{3}n^3+\frac{1}{2}n^2-\frac{7}{6}n.
        \end{gathered}
    \end{equation}
    Число необходимых арифметических действий в обратном ходе
    \begin{equation}
        Q_{\text{backward}}=n^2,
    \end{equation}

    Таким образом общее количество арифметических операций при решении СЛАУ методом Гаусса будет
    \begin{equation}
        Q=Q_{\text{forward}}+Q_{\text{backward}}=\frac{2}{3}n^3+\frac{3}{2}n^2-\frac{7}{6}n\approx\frac{2}{3}n^3.
    \end{equation}

    \subsection*{Вычисление определителей}

    Определитель матрицы при делении ее строки на ведущий элемент также делится на этот элемент, а при вычитании строки, умноженной на любое число, как известно, определитель новой матрицы равен определителю старой.
    Если каждую строку треугольной матрицы $\bold{A}^{\left(n-1\right)}$, полученной в ходе прямого хода метода Гаусса, разделить на соответствующий ведущий элемент $a_{{ii}}^{\left(i-1\right)}$, то ее определитель равен 1. Он получен в результате деления определителя $\det\left(\bold{A}\right)$ исходной матрицы $\bold{A}=\bold{A}^{\left(0\right)}$ системы~(\ref{7.1:f1}) на ведущие элементы на каждом шаге: $a_{11}^{\left(0\right)}$, $a_{22}^{\left(1\right)}$, $\ldots$, $a_{nn}^{\left(n-1\right)}$. Следовательно, если ни один из ведущих элементов не равен нулю, то
    \begin{equation}
        \det\left(\bold{A}\right)
        =a_{11}^{\left(0\right)}\cdot a_{22}^{\left(1\right)}\cdot\ldots\cdot a_{nn}^{\left(n-1\right)}.
    \end{equation}

    Таким образом попутно при решении системы методом Гаусса можно вычислить определитель матрицы системы линейных алгебраических уравнений.

    \subsection*{Выбор главного элемента}

    Во избежание деления на малый ведущий элемент рекомендуется осуществлять выбор наибольшего по модулю элемента и считать его ведущим (подробнее см. в \cite{lebedeva}).

    \subsection*{Обращение матриц}
    Для нахождения матрицы $\bold{A}^{-1}$ формально требуется решить $n$ систем, векторы правых частей которых представляют собой единичные орты. Однако, чтобы не осуществлять прямой ход $n$ раз, можно все преобразования делать в расширенной матрице размера $n \times 2n$ вида $\left(\bold{A},\bold{E}\right)$, просто расширив циклы по $j$ до значения $2n$ включительно.

    \subsection*{Задание}
	
	\begin{enumerate}
	    \item Разработать программу, которая реализует решение СЛАУ методом Гаусса, в том числе выполнить:
	    
	    \begin{itemize}
	        \item предварительную оценку системы (сделать выводы),
	        \item реализовать в методе выбор главного элемента: по столбцу, по строке, по столбцу и строке;
	        \item посчитать количество операций, и сравнить с тем, что указано в теоретической справке,
	        \item найти определитель матрицы коэффициентов СЛАУ.
	    \end{itemize}
	        
	    \item Рассмотреть систему $\bold{Ax}=\bold{b}$, в которой
	    \begin{equation}
	        \bold{A}_{ij}=\frac{1}{i+j-1},\quad i,j=1,2,\ldots,n.
	    \end{equation}
	    Положить точное решение этой СЛАУ
	    \begin{equation}
	        \bold{x}=\begin{pmatrix}
	            x_1 \\
	            x_2 \\
	             \vdots \\
	            x_n \\
	        \end{pmatrix}=
	        \begin{pmatrix}
	            1 \\
	            1 \\
	             \vdots \\
	             1
	        \end{pmatrix},
	    \end{equation}
	    из него найти правую часть $\bold{b}$. С учетом найденной правой части попробовать решить получившуюся СЛАУ относительно неизвестного вектора $\bold{x}$. Провести эксперимент при различных $n$.
	\end{enumerate}
	
    Программу рекомендуется разрабатывать на языках C/С++, python, можно использовать и любой другой язык, или даже математический пакет, но предпочтительнее будут первые варианты.

    \renewcommand{\bibname}{{Список литературы}}
    \bibliographystyle{gost2008}
    \begin{thebibliography}{3}
        \bibitem{lebedeva}
        Лебедева А. В., Пакулина А. Н. Практикум по методам вычислений. Часть 1. Учебно-методическое пособие. СПб.: Санкт-Петербургский государственный университет, 2021.~156 с.

        \bibitem{Mis}
        Мысовских И. П. Лекции по методам вычислений: Учебное пособие. СПб.: Издательство Санкт-Петербургского университета, 1998. 472~с.
        
        \bibitem{Fadeev}
        Фаддеев Д. К., Фаддеева В. Н. Вычислительные методы линейной алгебры. М.:  Государственное издательство физико-математической литературы, 1960. 655 с.
        
    \end{thebibliography}
\end{document}
